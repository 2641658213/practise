\documentclass[UTF8]{ctexart}
\usepackage{graphicx}
%这个是导入图片
\usepackage{subcaption}
\usepackage{amsmath}
%opening
\title{二121}
\author{zhs}
\date{\today}
\begin{document}
\maketitle

\begin{abstract}
	这里是摘要
\end{abstract}

hello,this is a example涵和时代价值,分享学习成果和感悟,鼓励团员和青年在实际工作和学习开中践行五四精

\textbf{这里是加粗}
\textit{这里是斜体qweqe}

一行回车是空格
\underline{这里是下划线,}

\section{这是第一个章节}


\subsection{这是一个次标题}


讨论五四精神的内涵和时代价值,分享学习成果和感悟,鼓励团员和青年在实际工作和学习开中践行五四精神。
\section{这是第二个章节}

\begin{table}[htbp]
\center
	\begin{tabular}
		{p{3cm} c| c}
		\hline\hline
		表格有三列,每列居中对齐&12&12\\
		45&56&78\\	
	%p 自定义列宽
	\end{tabular}
		\caption{标题}
\end{table}

爱因斯坦发现只能方程$E=mc^2$
%行内公式写在$之间$
\[
E=mc^2 
\]如果要写在中间


直接跟在中间公式格式的下面的文字是顶头的


\section{}
$a={k\varphi(n)+1}\over e$
{如果要单独的修改字体大小\fontsize{20}{12}\selectfont abcd}
%注意用{括起来}

在第 \ref{fig:myphoto} 章中我们介绍了...

详见第 \pageref{fig:myphoto} 页。
\begin{equation}
	\Delta u_{i,t} = \sum_{j=1}^{p} \alpha_{i,j} \Delta u_{i,t-j} + \varepsilon_{i,t}  \tag{2.1} 
\end{equation}
this is a good example to show a gongshi 

\begin{equation}
	\Delta u_{i,t} = \sum_{j=1}^{p} \alpha_{i,j} \Delta u_{i,t-j} + \varepsilon_{i,t}   \label{eq:example}
\end{equation}
\begin{figure}[p]
	\centering
	\includegraphics[width=3cm]{xh}
	\quad	\includegraphics[width=0.5\textwidth]{xh}
	\caption{图片标题}
	
	\label{fig:myphoto}
\end{figure}

\begin{figure}
	\centering
	\begin{subfigure}{0.3\textwidth}
		\centering
		\includegraphics[width=1\textwidth]{example-image-a}
		\caption{图1题目}
		\label{fig:subfig1}
	\end{subfigure}
	\hspace{0.2\textwidth} % 适当的间距
	\begin{subfigure}{0.3\textwidth}
		\centering
		\includegraphics[width=\textwidth]{example-image-b}
		\caption{图2题目}
		\label{fig:subfig2}
	\end{subfigure}
	
	\medskip % 适当的垂直间距
	
	\begin{subfigure}{0.4\textwidth}
		\centering
		\includegraphics[width=\textwidth]{example-image-c}
		\caption{图3题目}
		\label{fig:subfig3}
	\end{subfigure}
	\hspace{0.1\textwidth} % 适当的间距
	\begin{subfigure}{0.4\textwidth}
		\centering
		\includegraphics[width=\textwidth]{example-image}
		\caption{图4题目}
		\label{fig:subfig4}
	\end{subfigure}
	
	\caption{总标题}
	\label{fig:total}
\end{figure}

\end{document}
